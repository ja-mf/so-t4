\documentclass[a4paper,11pt]{article}
%\usepackage{ucs}
\usepackage{pstricks}
\usepackage[utf8]{inputenc}
\usepackage{amsmath}
\usepackage{amsfonts}
\usepackage{amssymb}
\usepackage{amsthm}
\usepackage[spanish]{babel}
\usepackage{graphicx}
\usepackage{graphics}
\usepackage[margin=1in]{geometry}
\usepackage{fancyhdr}
%\usepackage{subfigure}
%\usepackage{wrapfig}
%\usepackage[amssymb,mediumspace]{SIunits}
\usepackage{listings}
%\usepackage{fullpage}


\newcommand{\HRule}{\rule{\linewidth}{0.5mm}}
\pagestyle{fancy}
% Title Page
\setlength{\headwidth}{\textwidth}
\fancyhead[L]{ELO-321}% empty left
\fancyhead[R]{Teoria de Sistemas Operativos}

%\renewcommand{\headheight}{0.6in}


% listings

\definecolor{Brown}{cmyk}{0,0.81,1,0.60}
\definecolor{OliveGreen}{cmyk}{0.64,0,0.95,0.40}
\definecolor{CadetBlue}{cmyk}{0.62,0.57,0.23,0}
\definecolor{lightlightgray}{gray}{0.9}

\lstset{
%	language=C,                             % Code langugage
	basicstyle=\ttfamily,                   % Code font, Examples: \footnotesize, \ttfamily
	keywordstyle=\color{OliveGreen},        % Keywords font ('*' = uppercase)
	commentstyle=\color{gray},              % Comments font
%	numbers=left,                           % Line nums position
%	numberstyle=\tiny,                      % Line-numbers fonts
	stepnumber=1,                           % Step between two line-numbers
	numbersep=5pt,                          % How far are line-numbers from code
	backgroundcolor=\color{lightlightgray}, % Choose background color
	frame=none,                             % A frame around the code
	tabsize=2,                              % Default tab size
	captionpos=b,                           % Caption-position = bottom
	breaklines=true,                        % Automatic line breaking?
	breakatwhitespace=false,                % Automatic breaks only at whitespace?
	showspaces=false,                       % Dont make spaces visible
	showtabs=false,                         % Dont make tabls visible
	columns=flexible,                       % Column format
	showstringspaces=false,
}


\title{Teoría de Sistemas Operativos\\Tarea \#4}
\author{Juan Mucarquer\\2830027-1 \and Fernando Mora\\2830022-0 \and Adan Morales\\2830042-5}
\begin{document}

%\begin{titlepage}
%	\input{portada}
%\end{titlepage}
\maketitle
\renewcommand\thesubsubsection{\thesubsection\alph{subsubsection}}
\setcounter{section}{4}
%\section{Programacion \textsc{Shell}, Procesos y Hebras}
\subsection{Simulador \texttt{moss}}
\subsubsection{Archivos}
Al compilar mediante \texttt{javac}, se generaron todos los bytecode para las clases
existentes. Luego, ejecutando el programa con \texttt{java MemoryManagement commands memory.conf},
fue creado el archivo \texttt{tracefile}, el cual contiene un log con todas las operaciones hechas
en la simulación.

\subsubsection{Configuracion} Por omisión, se administran 64 páginas, tanto para memoria
virtual como fisica. Eso puede ser modificado con \texttt{numpages}, dentro de \texttt{memory.conf}

\subsubsection{\texttt{PageFault.java}} En este archivo está el algoritmo de reemplazo de página,
el cual ordena la lista y saca al primer elemento, siendo un algoritmo FIFO.

\end{document}
